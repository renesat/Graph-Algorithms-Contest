% Created 2020-10-11 Вс 14:49
% Intended LaTeX compiler: xelatex
\documentclass{extarticle}
                

\usepackage{longtable}
\usepackage{wrapfig}
\usepackage{rotating}
\usepackage[normalem]{ulem}
\usepackage{amsmath}
\usepackage{breqn}
\usepackage{textcomp}
\usepackage{amssymb}
\usepackage{capt-of}
\usepackage{hyperref}
\usepackage{minted}
\usepackage{polyglossia}
\setmainlanguage{russian}
\setotherlanguage{english}
\setkeys{russian}{babelshorthands=true}
\PolyglossiaSetup{russian}{indentfirst=true}
\usepackage{fontspec}
\setmainfont{Liberation Serif}
\usepackage{minted}
\usepackage[left=4cm,right=4cm, top=2cm,bottom=2cm,bindingoffset=0cm]{geometry}
\usepackage{xcolor}
\PassOptionsToPackage{final}{graphicx}
\usepackage{caption}
\usepackage{subcaption}
\usepackage{wrapfig}
\usepackage{array}
\definecolor{friendlybg}{HTML}{f0f0f0}
\date{}
\title{Задачи на поиск в глубину}
\hypersetup{
 pdfauthor={},
 pdftitle={Задачи на поиск в глубину},
 pdfkeywords={},
 pdfsubject={},
 pdfcreator={Emacs 27.1 (Org mode 9.4)}, 
 pdflang={Russian}}
\begin{document}



\section*{Задача 1}
\label{sec:orgb62fe60}
\subsection*{Постановка}
\label{sec:org37954e9}

В небольшой стране \(n\) городов.
Каждый город имеет номер — целое число от \(1\) до \(n\).
Столица имеет номер \(g_{1}\).
Дороги между городами двухсторонние, причем
есть только один путь от столицы до каждого города.

Карта хранится в следующем виде:
для каждого не столичного города \(i\) хранится число \(r_{i}\) -
номер последнего города на пути из столицы в город \(i\).

Было решено перенести столицу из города \(g_{1}\) в город \(g_{2}\).
После этого старое представление карты перестало быть верным.
Необходимо найти новое представление карты дорог в описанном выше виде.

\subsection*{Входные данные}
\label{sec:orgc51833b}

Первая строка содержит следующие 3 числа:
\(n\), \(g_{1}\), \(g_{2}\),
ограниченные следующими условиями
\(2 \leq n \leq 5 \cdot 10^{4}\) и \(1 \leq g_{1} \neq g_{2} \leq n\).
количество городов,
номер старой столицы и номер новой столицы соответственно.

Следующая строка содержит \(n-1\) чисел - старое представление карты дорог.

Для всех городов за исключением \(g_{1}\) задано целое число \(p_{i}\)
(номер последнего города на пути из столицы в город \(i\)).
Все города описаны в порядке увеличения номеров.

\subsection*{Выходные данные}
\label{sec:org91cd1c2}

Выведите \(n - 1\) чисел — новое представление карты дорог в том же формате.

\subsection*{Пример 1}
\label{sec:org1b720b0}

\begin{table}[H]
\begin{center}
\begin{tabular}{|m{4cm}|m{4cm}|}
\hline
Входные данные & Выходные данные \\ \hline
3 2 3

2 2
&
2 3
\\ \hline
\end{tabular}
\end{center}
\end{table}

\subsection*{Пример 2}
\label{sec:org2aeecb4}

\begin{table}[H]
\begin{center}
\begin{tabular}{|m{4cm}|m{4cm}|}
\hline
Входные данные & Выходные данные \\ \hline
6 2 4

6 1 2 4 2
&
6 4 1 4 2
\\ \hline
\end{tabular}
\end{center}
\end{table}

\pagebreak
\section*{Задача 2}
\label{sec:orgef181bd}
\subsection*{Постановка}
\label{sec:orgad8a20e}

Имеется \(n\) людей. Они общаются в \(m\) группах.
Человек \(x\) узнает новость из внешнего источника.
Затем этот пользователь отправляет новость всем своим друзьям
(друзья если оба общаются в какой-нибудь группе).
Друзья сообщают новость своим друзьям и тд.
Это происходит до того, как не останется пары друзе, в которой один
знает новость, а другой - нет.

Для каждого пользователя необходимо определить сколько пользователей узнает
новость, если он начнет её распространять.

\subsection*{Входные данные}
\label{sec:orgb72ba50}

В первой строке записаны два целых числа \(n\) и \(m\)
(\(1 \leq n\), \(m \leq 5 \cdot 10^{5}\)) — количество пользователей и групп, соответственно.

Далее следуют \(m\) строк с описанием групп.
Строка \(i\) начинается целым числом \(0 \leq g_{i} \leq n\) — количество пользователей в
группе \(i\).
Далее следуют \(g_{i}\) чисел, обозначающих пользователей.
\(\sum_{i=1}^{m} k_{i} \leq 5 \cdot 10^{5}\).

\subsection*{Выходные данные}
\label{sec:orgf9da829}

Выведите \(n\) целых чисел
равных количеству узнавших новость для каждого человека.

\subsection*{Пример}
\label{sec:orgd7d348d}

\begin{table}[H]
\begin{center}
\begin{tabular}{|m{4cm}|m{4cm}|}
\hline
Входные данные & Выходные данные \\ \hline
7 5

3 2 5 4

0

2 1 2

1 1

2 6 7
&
4 4 1 4 4 2 2
\\ \hline
\end{tabular}
\end{center}
\end{table}

\pagebreak
\section*{Задача 3}
\label{sec:org570b899}
\subsection*{Постановка}
\label{sec:orga2b5149}

В старом доме Антона был определён план
расположения комнат и коридоров между ними.
Коридоры двусторонние.
Комнаты пронумерованы от \(1\) до \(n\).

Антон хочет, чтобы новый дом выглядел также как и предыдущий.
Для этого в нем должно быть \(n\) комнат и если существовал
коридор из \(i\) в \(j\), то он есть в новом доме.

Антон строит дом так, что он начинает строить коридор из
некоторой комнаты и пробивает их до тех пор, пока
не получит все коридоры и вернётся в начальную комнату.

Также известно, что Антон строит, не прерываясь, то есть пока не
закончит строительство. По уже построенным коридорам он не ходит.

Антону скучно строить коридоры в одном порядке. Поэтому он,
зная порядок построения коридоров в предыдущем доме, хочет
построить коридоры в другом порядке.
Этот порядок представляет собой список комнат в процессе их
посещения.
Новый список должен быть лексикографически наименьший,
но строго больше предыдущего.

\subsection*{Входные данные}
\label{sec:orgeb4908d}

В первой строке - два целых числа \(n\) и \(m\)
(\(3 \leq n \leq 10^{2}\), \(3 \leq m \leq 2\cdot 10^{3}\)) — количество комнат и коридоров в
доме Антона.
В следующей строке записано \(m + 1\) чисел, не превышающих \(n\):
описание старого маршрута в виде списка комнат,
которые он посещал.
Гарантируется, что последнее число в этом списке
совпадает с первым.

Первая комната - это главный вход, поэтому Антон всегда
должен начинать строить именно с неё.

Можете предполагать, что ни одна комната не соединена сама
с собой коридором, и если существует коридор между
некоторой парой комнат, то только один.
В то же время, могут существовать изолированные комнаты,
не соединённые коридорами вообще.

\subsection*{Выходные данные}
\label{sec:orged795e8}

Выведите \(m + 1\) чисел, не превышающих \(n\):
описание нового маршрута, в соответствии с которым он должен
построить новый дом.
Если такого маршрута не существует, выведите \textbf{None}.

\subsection*{Пример 1}
\label{sec:org6a26c04}

\begin{table}[H]
\begin{center}
\begin{tabular}{|m{4cm}|m{4cm}|}
\hline
Входные данные & Выходные данные \\ \hline
3 3

1 2 3 1
&
1 3 2 1
\\ \hline
\end{tabular}
\end{center}
\end{table}

\subsection*{Пример 2}
\label{sec:orge96f7c4}

\begin{table}[H]
\begin{center}
\begin{tabular}{|m{4cm}|m{4cm}|}
\hline
Входные данные & Выходные данные \\ \hline
3 3

1 3 2 1
&
None
\\ \hline
\end{tabular}
\end{center}
\end{table}

\pagebreak
\section*{Задача 4}
\label{sec:orgb1f46a6}
\subsection*{Постановка}
\label{sec:orge854c50}

Дан неориентированный граф из \(n\) вершин
и \(m\) ребер.
Вершины пронумерованы целыми числами от \(1\) до \(n\).

\textbf{Граф гармоничный} если для каждой тройки целых чисел \((l,m,r)\), где
\(1 \leq l < m < r \leq n\),
если есть путь из вершины \(l\) в вершину \(r\),
тогда существует путь из вершины \(l\) в вершину \(m\).

Тоесть, в гармоничном графе, если из вершины \(l\)
можно по ребрам дойти до вершины
\(r\) (\(l<r\)), тогда также должно быть можно
дойти до вершин \((l+1),(l+2),\dots,(r−1)\).

Найдите минимальное число ребер которых надо добавить в граф,
чтобы он стал гармоничным.

\subsection*{Входные данные}
\label{sec:org2e05f42}

В первой строке - два целых числа
\(n\) и \(m\) (\(3 \leq n \leq 2 \cdot 10^{5}\) и \(1 \leq m \leq 2 \cdot 10^{5}a\)).

В следующих \(m\) строках записаны по два целых числа
\(t_{i}\) и \(g_{i}\) (\(1 \leq t_{i}, g_{i} \leq n\), \(t_{i} \neq g_{i}\)),
описывающих ребро между вершинами \(t\) и \(g\).

Граф простой (без петель и между каждой парой вершин не более одного ребра).

\subsection*{Выходные данные}
\label{sec:org1ab7414}

Минимальное количество ребер которое необходимо добавить в граф.

\subsection*{Пример}
\label{sec:org25482f8}

\begin{table}[H]
\begin{center}
\begin{tabular}{|m{4cm}|m{4cm}|}
\hline
Входные данные & Выходные данные \\ \hline
14 8

1 2

2 7

3 4

6 3

5 7

3 8

6 8

11 12
&
1
\\ \hline
\end{tabular}
\end{center}
\end{table}


\pagebreak

\section*{Задача 5}
\label{sec:org3bacd52}
\subsection*{Постановка}
\label{sec:org30bb0a3}

Турнир — ориентированный граф без петель, в котором каждая
пара вершин соединена ровно одним ребром.
Для любых двух вершин \(u\) и \(v\) (\(u \neq v\)) либо есть ребро из
\(u\) в \(v\), либо есть ребро из \(v\) в \(u\).

Дан турнир из \(n\) вершин. Требуется найти в нем цикл длины три.

\subsection*{Входные данные}
\label{sec:orge2c6641}

В первой строке задано целое число \(n\) (\(1 \leq n \leq 5000\)).
В следующих \(n\) строках задана матрица смежности графа \(G\).
\(A_{ij}=1\) если есть ребро из \(i\) в \(j\), в противном случае ребра нет.

\subsection*{Выходные данные}
\label{sec:org75f0d83}

Выведите 3 номера вершин цикла если он есть. Если цикл длины 3 отсутствует, то
выведите \textbf{None}.
Если решений несколько, выведите любое.

\subsection*{Пример 1}
\label{sec:org2fcdd98}

\begin{table}[H]
\begin{center}
\begin{tabular}{|m{4cm}|m{4cm}|}
\hline
Входные данные & Выходные данные \\ \hline
5

00100

10000

01001

11101

11000
&
1 3 2
\\ \hline
\end{tabular}
\end{center}
\end{table}

\subsection*{Пример 2}
\label{sec:org96d1f76}

\begin{table}[H]
\begin{center}
\begin{tabular}{|m{4cm}|m{4cm}|}
\hline
Входные данные & Выходные данные \\ \hline
5

01111

00000

01000

01100

01110
&
None
\\ \hline
\end{tabular}
\end{center}
\end{table}

\pagebreak
\end{document}
